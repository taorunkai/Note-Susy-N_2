\chapter{Solution of Some Exercises}

%=============chapter One Exercise Solution===========

 \section{Exercises Solution for Chapter One}
 
 \noindent\textbf{Solution \ref{Exe:Lorentz transformation of sigma matrice}}
 
 Using \eqref{infinitesimal transformation for two component spinor} \eqref{expansion of vector representation} and 
 \eqref{explicit form of vector representation} to write the infinitesimal form to first order for both sides of 
 \begin{equation}
 s\s^{\m}s^{\dagger} = {\Lambda_\n}^\m \s^\n
 \end{equation} 
 LHS is  
 \begin{align}
 s\s^{\m}s^{\dagger} = \bigg[1+\half(\half i \e_{ijk}\w_{ij}+\w_{k0})\s_k\bigg]\s^{\m}\bigg[1+\half(-i\half \e_{ijk}\w_{ij}+\w_{k0})\s_k\bigg]
 \end{align}
 RHS is
 \begin{align}
 {\Lambda_\n}^\m \s^\n =& (\de_\n^\m + \frac{i}{2}\w_{\a\be}{((J^\text{vec})^{\a\be})_\n}^\m)\s^{\n}
 \notag\\
 =& \bigg[\de_\n^\m + 
 i\w_{0i}{((J^\text{vec})^{0i})_\n}^\m+
 \frac{i}{2}\w_{ij}{((J^\text{vec})^{ij})_\n}^\m
 \bigg]\s^{\n}\notag\\
 =& \bigg[\de_\n^\m + 
 i\w_{0i}[i(\de^0_\n g^{i\m}-\de^i_\n g^{0\m})]+
 \frac{i}{2}\w_{ij}[-i(\de^i_\nu g^{j\m}-\de^j_\n g^{i\m})]
 \bigg]\s^{\n}
 \end{align}
 For $\mu = 0$, LHS is
 \begin{align}
  s\s^{0}s^{\dagger} &= \bigg[1+\half(\half i \e_{ijk}\w_{ij}+\w_{k0})\s_k\bigg]\s^{0}\bigg[1+\half(-i\half \e_{ijk}\w_{ij}+\w_{k0})\s_k\bigg]\notag\\
  &= \s^0 
  +\frac{1}{4}i\e_{ijk}\w_{ij}\s_k
  -\frac{1}{4}i\e_{ijk}\w_{ij}\s_k
  +\half\w_{k0}\s_k+\half\w_{k0}\s_k
  \notag\\
  &=\s^0 +\w_{k0}\s_k
 \end{align}
 RHS is
 \begin{align}
   {\Lambda_\n}^0 \s^\n 
   =& \bigg[\de_\n^0 + 
   i\w_{0i}[i(\de^0_\n g^{i0}-\de^i_\n g^{00})]+
   \frac{i}{2}\w_{ij}[-i(\de^i_\nu g^{j0}-\de^j_\n g^{i0})]
   \bigg]\s^{\n}\notag\\
   =& \s^0 +i\w_{0i}i\s_{i}=\s^{0}+\w_{i0}\s_i
 \end{align}
 So for $\m = 0$, left and right sides are the same. 
 
 For $\m = l$ with $l=1,2,3$. The LHS is 
 \begin{align}
 s\s^{l}s^{\dagger} &= \bigg[1+\half(\half i \e_{ijk}\w_{ij}+\w_{k0})\s_k\bigg]\s^{l}\bigg[1+\half(-i\half \e_{ijk}\w_{ij}+\w_{k0})\s_k\bigg]\notag\\
 & =\s^{l} i\frac{1}{4}\e_{ijk}\w_{ij}\comm{\s_k}{\s_l}
 + \half \w_{k0}\acomm{\s_k}{\s_l}\notag\\
 & = \s^{l}
 +\frac{1}{4}i\e_{ijk}\w_{ij}2i\e_{klm}\s_m
 +\frac{1}{2}\w_{k0}2\de_{kl}\notag\\
 &=\s^{l}
 - \half(\de_{il}\de_{jm}-\de_{im}\de_{jl})\s_m\w_{ij}
 +\w_{l0}\notag\\
 &= -\s_j\w_{lj} + \w_{l0}+\s_l
 \end{align}
 RHS is
 \begin{align}
 {\Lambda_\n}^l \s^\n
 &=
 \bigg[\de_\n^l + 
 i\w_{0i}[i(\de^0_\n g^{il}-\de^i_\n g^{0l})]+
 \frac{i}{2}\w_{ij}[-i(\de^i_\nu g^{jl}-\de^j_\n g^{il})]
 \bigg]\s^{\n}\notag\\
 &=
 \s_l -\w_{0l}+\half \w_{il}\s_i -\half\w_{lj}\s_j
 \notag\\
 &=
 \s_l -\w_{lj}\s_j -\w_{0l}
 \end{align}
 So for $\m = l$, left and right sides are the same. So we have verified \eqref{Lorentz transformation of sigma matrice}.
 
 \noindent\textbf{Solution \ref{Exe: product of two sL}}
 
we begin from \eqref{two component product of Majorana spinor}
\begin{equation}
\tht_{\a}\tht_{\be} = 
\underbrace{\frac{1}{4} (\e\g^5)_{\a\be} (\tht^{\tra}\e\g^5\tht)}_{\textcircled{1}}+ 
\underbrace{\frac{1}{4}(\g_{\m}\e)_{\a\be}({\tht}^{\tra}\e\g^{\m}\tht)}_{\textcircled{2}}+
\underbrace{\frac{1}{4}(\e)_{\a\be}(\tht^{\tra}\e\tht)}_{\textcircled{3}}
\notag
\end{equation} 
Write the product of two left-hand Majorana spinors
in matrix form
\begin{equation}
(\tht_L)_{\a}(\tht_L)_{\be} 
=(\tht_L\tht_L^{\tra})_{\a\be} 
=(\half(1+\g^5)\tht)_{\a}
 (\tht^{\tra}\half(1+\g^5))_{\be} 
\end{equation} 
We can multiply the projective operator in the same way from the right side of \eqref{two component product of Majorana spinor}. 
\begin{align}
(\half(1+\g^5))\,\textcircled{1}\,(\half(1+\g^5)) 
& = \frac{1}{16} (\e\g^5+\e)(1+\g^5)
(\tht^{\tra}\e\g^5\tht)
\notag \\
& = \frac{1}{8}\e(1+\g^5)
(\tht^{\tra}\e\g^5\tht) \notag\\
(\half(1+\g^5))\,\textcircled{2}\,(\half(1+\g^5)) 
& = \frac{1}{16}
(\g_\m\e + \g_5\g_\m\e)(1+\g^5)
(\tht^{\tra}\e\g^{\m}\tht)
\notag\\
& =  \frac{1}{16}
(\g_\m\e + \g^5\g_\m\e + \g_\m\e\g^5 + \g^5\g_\m\e\g^5)
(\tht^{\tra}\e\g^{\m}\tht)
\notag\\
& = 0 \notag\\
(\half(1+\g^5))\,\textcircled{3}\,(\half(1+\g^5)) 
&  = \frac{1}{8}
(1+\g^5)\e 
(\tht^{\tra}\e\tht) \notag
\end{align}
Thus
\begin{equation}
(\tht_L)_{\a}(\tht_L)_{\be} =
\frac{1}{4} [\e(1+\g^5)]_{\a\be}
(\tht^{\tra}\frac{1+\g^5}{2}\tht)=
\frac{1}{4} [\e(1+\g^5)]_{\a\be}
(\tht_L^{\tra}\tht_L)
\end{equation}

%=============Chapter Two Exercise Solution===============
\section{Exercise Solution for Chapter Two}

\textbf{Solution \ref{Exe:Invariant tensor C}}

Acting the Lorentz transformation on both sides of  \eqref{Internal algebra relation}
\begin{equation}
U^{-1}(\Lambda)\comm{B_\a}{B_\be}U(\Lambda)
= i C^{\g}_{\a\be}{D_{\g}}^{\g'}(\Lambda)B_{\g'}
\end{equation}
The left side is
\begin{align}
U^{-1}(\Lambda)\comm{B_\a}{B_\be}U(\Lambda)
&=\comm{{D_\a}^{\a'}(\Lambda)B_{\a'}}
{{D_\be}^{\be'}(\Lambda)B_{\be'}}\notag\\
&={D_\a}^{\a'}(\Lambda){D_\be}^{\be'}(\Lambda)
iC^{\g'}_{\a'\be'}B_{\g'}
\end{align}
Compare both sides we have
\begin{equation}
C^{\g}_{\a\be}{D_{\g}}^{\g'}(\Lambda)={D_\a}^{\a'}(\Lambda){D_\be}^{\be'}(\Lambda)
\,C^{\g'}_{\a'\be'}
\end{equation}
Act ${D_{\g'}}^{\de}(\Lambda^{-1})$ on both sides, we have
\begin{equation}
C^{\g}_{\a\be}={D_\a}^{\a'}(\Lambda){D_\be}^{\be'}(\Lambda)
{D_{\g'}}^{\g}(\Lambda^{-1})\,C^{\g'}_{\a'\be'}
\end{equation}
Where we have used Einstein summation convention in calculation.

\qquad

\noindent\textbf{Solution \ref{Exe:property of lie algebra matrix}}

Act $C^{\a}_{\g\de}$ on both sides of \eqref{property of lie algebra matrix}
\begin{align}
  C^{\g}_{\a\be}C^{\a}_{\g\de}&=
  {D_{\a}}^{\a'}(\Lambda)
  {D_{\be}}^{\be'}(\Lambda)
  {D_{\g'}}^{\g}(\Lambda^{-1})
  C^{\g'}_{\a'\be'}
  {D_{\a''}}^{\a}(\Lambda^{-1})
  {D_{\g}}^{\g''}(\Lambda)
  {D_{\de}}^{\de''}(\Lambda)C^{\a''}_{\de''\g''}\notag\\
  &=
  \de_{\a''}^{\a'}
  \de_{\g'}^{\g''}
  {D_{\be}}^{\be'}(\Lambda)
  {D_{\de}}^{\de''}(\Lambda)
  C^{\g'}_{\a'\be'}
  C^{\a''}_{\de''\g''}\notag\\
  &=
  {D_{\be}}^{\be'}(\Lambda)
  {D_{\de}}^{\de'}(\Lambda)
  C^{\g''}_{\a''\be'}
  C^{\a''}_{\de'\g''}
\end{align}
Using the definition of Lie algebra matrice \eqref{definition of lie algebra metric}, we immedietely obtain 
\begin{equation}
  g_{\be\de} = 
  {D_{\be}}^{\be'}(\Lambda)
  {D_{\de}}^{\de'}(\Lambda)
  g_{\be'\de'}
\end{equation}
Where we have used Einstein summation convention in calculation.