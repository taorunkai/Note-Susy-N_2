\chapter{4D Lorentz Symmetry}

This note is written as a first look of supersymmetric theory for readers who has a brief idea
about quantum field theory. To make this note more self-consistent both in contents and conventions,
we make the first chapter a review of Lorentz symmetry. Readers who are familiar with this topic
can skip it safely.

\section{Lorentz transformation}

All the high energy physics theory we are talking about need to satisfy Lorentz invariance. So
Lorentz invariance turns out to be a starting point for a physics theory, which we are going to
build.

Lorentz transformation is defined to make the spacetime interval invariant. The spacetime interval
$\dd{s}$ is defined as
\begin{equation}
  \dd{s}^2 = (\dd{x}^0)^2-(\dd{x}^1)^2-(\dd{x}^2)^2-(\dd{x}^3)^2.
\end{equation}
It is convenient to write it in the covariant form
\begin{equation}
  \dd{s}^2 = g_{\mu\nu}\dd{x}^{\mu}\dd{x}^{\nu}.
\end{equation}
Here, $g_{\mu\nu}$ is the \kwd{Minkowski metric} and (in our convention) is defined as
\begin{equation}
  g_{\mu\nu} = \diag(1, \, -1, \, -1, \, -1).
\end{equation}
A linear transformation
\begin{equation} \label{eq:lorentz-transformation}
  x' = \Lambda x + a
\end{equation}
that satisfies
\begin{equation} \label{eq:lorentz-invariance}
  \dd{s}'^2 = \dd{s}^2
\end{equation}
is called \kwd{Lorentz transformation}, or \kwd{Poincare transformation} in some books. You can
easily verify that all the Lorentz transformations form a group, which is called
\kwd{Lorentz group}. If $a=0$ in Eq.~\eqref{eq:lorentz-transformation}, then those transformations
form a subgroup called \kwd{homogeneous Lorentz group}. Eq.~\eqref{eq:lorentz-invariance} is called
\kwd{Lorentz invariance}.

% TODO: (stone-zeng, 2018-03-08) 看不懂
In a Lorentz invariant theory, coordinates as well as operators should have a transformation rule according to Lorentz transformation, so that we can construct Lorentz invariance formula. That is to say, we need to clarify which representation of Lorentz group \footnote{More especially the homogeneous Lorentz group ($a=0$), since representation of translation is somehow trivial.} they belong to.

To do this, first we need to write down the explicit form (a \kwd{vector representation}) of
Eq.~\eqref{eq:lorentz-transformation} with $a=0$. The boost transformation along $x^1$-direction is
then
\begin{equation} \label{eq:boost-explicit-form}
  \mqty({x^0}' \\ {x^1}' \\ {x^2}' \\{x^3}')
  = \mqty(\cosh \phi_1 & \sinh \phi_1 & 0 & 0 \\
          \sinh \phi_1 & \cosh \phi_1 & 0 & 0 \\
          0            & 0            & 1 & 0 \\
          0            & 0            & 0 & 1) \,
    \mqty(x^0 \\ x^1 \\ x^2 \\x^3)
  \defeq B_1 \mqty(x^0 \\ x^1 \\ x^2 \\x^3).
\end{equation}

The infinitesimal form of $\Lambda$ as
\begin{equation} \label{eq:Lambda-infinitesimal-form}
  \Lambda(\V{\theta}, \, \V{\phi})
  = 1 + \ii\V{K}^V\cdot\V{\phi} + \ii\V{J}^V\cdot\V{\theta} + \cdots
\end{equation}
where $\V{K}^V=(K_1^V, \, K_2^V, \, K_3^V)$, $\V{J}^V = (J_1^V, \, J_2^V, \, J_3^V)$ are generators
of boost and rotation in a vector representation, respectively. Using
Eq.~\eqref{eq:boost-explicit-form}, we have
\begin{equation} \label{eq:K1-vector-representation}
  K_1^V = \frac{1}{\ii}\pdv{B_1}{\f_1}\eval{}_{\f_1=0}
        = -\ii \, \mqty(0 & 1 & 0 & 0 \\
                        1 & 0 & 0 & 0 \\
                        0 & 0 & 0 & 0 \\
                        0 & 0 & 0 & 0)
\end{equation}
In the same way, all the generators of homogeneous Lorentz group in vector representation can be
obtained:
\begin{align}
  K_1^V &= -\ii \, \mqty(0 &  1 &  0 &  0  \\
                         1 &  0 &  0 &  0  \\
                         0 &  0 &  0 &  0  \\
                         0 &  0 &  0 &  0), &
  K_2^V &= -\ii \, \mqty(0 &  0 &  1 &  0  \\
                         0 &  0 &  0 &  0  \\
                         1 &  0 &  0 &  0  \\
                         0 &  0 &  0 &  0), &
  K_3^V &= -\ii \, \mqty(0 &  0 &  0 &  1  \\
                         0 &  0 &  0 &  0  \\
                         0 &  0 &  0 &  0  \\
                         1 &  0 &  0 &  0); \label{eq:K-vector-representation} \\
  J_1^V &= -\ii \, \mqty(0 &  0 &  0 &  0  \\
                         0 &  0 &  0 &  0  \\
                         0 &  0 &  0 &  1  \\
                         0 &  0 & -1 &  0), &
  J_2^V &= -\ii \, \mqty(0 &  0 &  0 &  0  \\
                         0 &  0 &  0 & -1  \\
                         0 &  0 &  0 &  0  \\
                         0 &  1 &  0 &  0), &
  J_3^V &= -\ii \, \mqty(0 &  0 &  0 &  0  \\
                         0 &  0 &  1 &  0  \\
                         0 & -1 &  0 &  0  \\
                         0 &  0 &  0 &  0). \label{eq:J-vector-representation}
\end{align}
% CHECK HERE ===================================================================
It can be found that $K_i^V$ are not hermitian matrices. On the other hand, Eq.~\eqref{eq:Lambda-infinitesimal-form} tells us that vector representation is not unitary. In fact, any representation of homogeneous Lorentz group is not unitary thus they can not be described as a state vector. This is one reason why we need second quantization for a field theory.

If calculating the commutation relations of \eqref{eq:K1-vector-representation}--\eqref{eq:J-vector-representation}, we can find that they form a closed Lie algebra:
\begin{align}
  [K_i^V,K_j^V] &= -i \e_{ijk} J_k^V, \label{eq:homogeneous-Lie-algebra-commutation-1} \\
  [J_i^V,K_j^V] &=  i \e_{ijk} K_k^V, \label{eq:homogeneous-Lie-algebra-commutation-2} \\
  [J_i^V,J_j^V] &=  i \e_{ijk} J_k^V  \label{eq:homogeneous-Lie-algebra-commutation-3}
\end{align}

The commutation relations define a Lie algebra and determine the group space of corresponding Lie group near the identity element. Thus the generators for any other representation of homogeneous Lorentz group will still satisfy \eqref{eq:homogeneous-Lie-algebra-commutation-1}--\eqref{eq:homogeneous-Lie-algebra-commutation-3}.
\begin{align}
  [K_i,K_j] &= -i \e_{ijk} J_k, \label{eq:arbitrary-homogeneous-Lie-algebra-commutation-1} \\
  [J_i,K_j] &=  i \e_{ijk} K_k, \label{eq:arbitrary-homogeneous-Lie-algebra-commutation-2} \\
  [J_i,J_j] &=  i \e_{ijk} J_k  \label{eq:arbitrary-homogeneous-Lie-algebra-commutation-3}
\end{align}
Conversely, \eqref{eq:arbitrary-homogeneous-Lie-algebra-commutation-1}--\eqref{eq:arbitrary-homogeneous-Lie-algebra-commutation-3} can be used to find other representation. A unified form of commutation relation is somehow more useful. Define antisymmetric operator $J_{\mu\nu}$ (where $\m,\n = 0,1,2,3$) as
\begin{equation} \label{eq:redefine-Lorentz-generator-operators}
  J_{ij} = \e_{ijk}J_k \quad (i,j = 1,2,3), \qquad
  J_{0i} = K_i \quad (i = 1,2,3)
\end{equation}
The commutation relation can then be written as
\begin{equation} \label{eq:Lorentz-operator-commutation-covariant-form}
  \comm{J_{\mu\nu}}{J_{\rh\s}} =
  i \, (  g_{\n\rh}J_{\m\s} - g_{\m\rh}J_{\n\s}
        + g_{\m\s}J_{\n\rh} - g_{\n\s}J_{\m\rh})
\end{equation}
which is in a covariant form.

\begin{Exe}
  Verify that the Lorentz transformation \eqref{eq:lorentz-transformation} forms a group.
\end{Exe}

\begin{Exe}
  Find the equivalence of \eqref{eq:arbitrary-homogeneous-Lie-algebra-commutation-1}, \eqref{eq:arbitrary-homogeneous-Lie-algebra-commutation-2} and \eqref{eq:arbitrary-homogeneous-Lie-algebra-commutation-3} with \eqref{eq:Lorentz-operator-commutation-covariant-form} through direct calculation by the redefinition of \eqref{eq:redefine-Lorentz-generator-operators}.
\end{Exe}

\section{Dirac matrices}

In this section we introduce an important class of representation which satisfy
\eqref{eq:Lorentz-operator-commutation-covariant-form} called spinor representation.
It is construct by four dimension \kwd{Clifford Algebra}
\footnote{\eqref{defination of Dirac matrice 1} is the exact definition of Clifford Algebra, but we still need \eqref{defination of Dirac matrice 2} to ensure the hermitian and antihermitian properties of generators. } called Dirac matrice, defined by
\begin{equation} \label{defination of Dirac matrice 1}
\acomm{\g^{\m}}{\g^{\n}} = 2g^{\mu\nu}
\end{equation}
\begin{equation} \label{defination of Dirac matrice 2}
{\g^{\mu}}^{\dagger} = \g^0\g^{\m}\g^0
\end{equation}
The matrice generator satisfied commutation relation \eqref{eq:Lorentz-operator-commutation-covariant-form} is obtained from the Dirac matrice by defining
\begin{equation} \label{definition of generators in Dirac representation}
\Si^{\mu\nu}\equiv \frac{i}{4}\comm{\g^{\m}}{\g^{\n}}
\end{equation}
using \eqref{defination of Dirac matrice 2}, we can obtain that $\Si^{ii}$ is hermitian while $\Si^{0i}$ is antihermitian, the same as generators in vector representation.

The acting space of such representation is a four dimension spinor (a complex vector).
\begin{equation}
  \Psi(x) = \mqty(\psi_1(x)\\ \psi_2(x)\\ \psi_3(x)\\ \psi_4(x))
\end{equation}
In analogue of \eqref{eq:Lambda-infinitesimal-form}, we write the infinitesimal form for group element now in covariant way, that is
\begin{equation} \label{the infinitesimal form of Sla}
S(\w) = 1+ \frac{i}{2}\omega_{\mu\nu}\Si^{\mu\nu}+\cdots
\end{equation}
$\omega_{\mu\nu}$ is a antisymmetry parameters. with the degree of freedom is the same as the Lorentz group. Compare with \eqref{eq:Lambda-infinitesimal-form} and \eqref{eq:redefine-Lorentz-generator-operators}, we can obtain the relations for the parameters.
\begin{equation} \label{parameter redefinition}
\tht_k = \half \w_{ij}\e_{ijk}, \quad \f_i = \w_{0i}
\end{equation}
Write the group element in finite form is
\begin{equation} \label{the finite form of Sla}
S(\w) = \exp{\frac{i}{2}\omega_{\mu\nu}\Si^{\mu\nu}}
\end{equation}
Now the Lorentz transformation for 4 dimension Dirac spinor  is
\begin{equation} \label{Lorentz transformation of Dirac spinor}
\Psi(x)\rightarrow\Psi'(x') = S(\la)\Psi(x)
\end{equation}
Since now the group element is not unitary as illustrate before, the dual space of $\Psi(x)$ can not be just complex conjugate. That is to say under Lorentz  transformation.
\begin{equation}
  \Psi^{\dagger} \Psi
\rightarrow
\Psi^{\dagger} S^{\dagger} S \, \Psi \neq \Psi^{\dagger} \Psi
\end{equation}
$\Psi(x)^{\dagger} \Psi(x) $ is not a scalar.
We need to define a dual spinor with respect to $\Psi(x)$.
\begin{equation} \label{definition of dual spinor}
\bar{\Psi}(x) = -\Psi(x)\,\g^0
\end{equation}
Using the property that $\Si^{ij}$s are hermitian, while $\Si^{0i}$s are antihermitian, we obtain
\begin{equation} \label{property of S(lambda)}
S^{\dagger} \g^0 = \g^0 S^{-1}
\end{equation}
Thus, the Lorentz transformation of $\bar{\Psi}$ can be obtained from \eqref{Lorentz transformation of Dirac spinor}  \eqref{definition of dual spinor}, and \eqref{property of S(lambda)}.
\begin{equation} \label{Lorentz transformation of barPsi}
\bar\Psi \rightarrow \bar{\Psi}' = \bar{\Psi}S^{-1}
\end{equation}
Thus $\bar{\Psi}\Psi$ is a scalar.
To define a chiral spinor, we need to introduce
\begin{equation}
  \g^5 = i\g^0\g^1\g^2\g^3
\end{equation}
Using the definition for Dirac matrice \eqref{defination of Dirac matrice 1} and \eqref{defination of Dirac matrice 2}, we can verify that
\begin{equation} \label{property of gamma5}
\g^5 =(\g^5)^{-1} = (\g^5)^{\dagger}
\end{equation}
A spinor is a \kwd{left-chiral spinor} $\Psi_L$ if
\begin{equation} \label{definition of left chiral spinor}
\g^5\Psi_L = \Psi_L
\end{equation}
on the other hand, a spinor is a \kwd{right-chiral spinor} $\Psi_R$ if
\begin{equation} \label{definition of right chiral spinor}
\g^5 \Psi_R = -\Psi_R
\end{equation}


Thus using the property \eqref{property of gamma5}, we can introduce projective operator to project arbitrary spinor to corresponding left and right chiral $\Psi_L = P_L\Psi$, $\Psi_R = P_R\Psi$ .
\begin{equation}
  P_L = \frac{1+\g^5}{2} ,\quad P_R = \frac{1-\g^5}{2}
\end{equation}

It is very useful to consider a particular representation for the Dirac matrices called \kwd{Weyl representation}.
\begin{equation} \label{Weyl representation}
\g^0 = \mqty(  0 & -1\\
-1 &  0),\quad
\g^i = \mqty(  0 &  \s^{i}\\
-\s^{i}&   0   ),\quad
\g^5 = \mqty(  1 &  0\\
0 & -1)
\end{equation}
where $\s^i$ are Pauli matrice.
\begin{equation} \label{Pauli matrice}
\s^1 = \mqty( 0 & 1 \\
1 & 0),\quad
\s^2 = \mqty( 0 & -i \\
i & 0),\quad
\s^3 = \mqty( 1 & 0 \\
0 & -1)
\end{equation}
You can check by straight calculation that \eqref{Weyl representation} satisfied the definition \eqref{defination of Dirac matrice 1} and \eqref{defination of Dirac matrice 2}. We present Weyl representation here since it provides a way to construct two dimension representation which we will talk about in next section.

I should mention that just like an arbitrary $2\times2$ complex matrice can be obtained by linear combination of three Pauli matrices and an Identity one, $4\times4$ complex matrices can also be constructed by following sixteen independent covariant matrices $1,\g^{\m},\Si^{\mu\nu},\g^{\m}\g^5,\g^5$, which is useful in later analysis. These matrice are orthogonal in the sense that for arbitary two matrice $M_i$ and $M_j$ in the bases.  
\begin{equation}
\text{tr}(M_i M_j) = C(M_i)\delta_{ij}
\end{equation}
where $C(M_i)$ can be found in \eqref{orthogonal property of 16 covariant matrices}.

After the brief introduction of Dirac representation, the generators relation between Dirac representation and vector representation are easily obtained by require $\bar{\Psi}\gamma^{\m}\Psi$ transform as a vector under Lorentz transformation.
\begin{equation} \label{Lorentz transformation of gamma matrice}
\g^{\m}{\Lambda_{\m}}^{\rh}=S^{-1}\g^{\rh}S
\end{equation}

\begin{Exe}
Verify that \eqref{definition of generators in Dirac representation} satisfy the commutation relation \eqref{eq:Lorentz-operator-commutation-covariant-form}.
\end{Exe}
\begin{Exe}
Verify \eqref{Lorentz transformation of barPsi}.
\end{Exe}

\begin{Exe}
Verify the spinors obtained by projective operator satisfy the definition of left and right chiral spinor \eqref{definition of left chiral spinor} and
  \eqref{definition of right chiral spinor}.
\end{Exe}

%=================two component representation=================%

\section{Two-component Weyl spinor}
This section we will obtain two-component Weyl spinor from four component one. Two-component representation is important in constructing supersymmetric algebra.

First we need to write \eqref{Weyl representation} in a unified form
\begin{equation} \label{unified form of gamma matrice}
\g^{\m} =\mqty( 0            & \s^{\m}\\
\bar{\s}^{\m}& 0     )
\end{equation}
where
\begin{equation}
  \s^{\m} = (\s^0,\s^i) = (-1,\s^{i}),\quad
\bar{\s}^{\m}=(\bar{\s}^0,\bar{\s}^i)=(-1,-\s^{i})
\end{equation}

Using the definition \eqref{definition of generators in Dirac representation}, we find
\begin{align}\label{definition of generators in Weyl representation}
\Si^{\mu\nu}
&= \frac{i}{4}
\mqty(
\s^{\m}\bar{\s}^{\n}-\s^{\n}\bar{\s}^{\m}&
0\\
0&
\bar{\s}^{\m}\s^{\n}-\bar{\s}^{\n}\s^{\m}) \notag\\
& = i \mqty(
\s^{\mu\nu}&
0\\
0&
\bar{\s}^{\mu\nu})
\end{align}
with the definition
\begin{equation}
  \s^{\mu\nu} =\frac{1}{4}(\s^{\m}\bar{\s}^{\n}-\s^{\n}\bar{\s}^{\m}),\quad
\bar{\s}^{\mu\nu}
=\frac{1}{4}(\bar{\s}^{\m}\s^{\n}-\bar{\s}^{\n}\s^{\m})
\end{equation}
The generator now is block diagonal, and so is the group element from\eqref{the finite form of Sla}
\begin{equation} \label{group element for weyl representation}
S(\w) = \mqty(
e^{-1/2\w_{\mu\nu}\s^{\mu\nu}} &
0                           \\
0                           &
e^{-1/2\w_{\mu\nu}\bar{\s}^{\mu\nu}})
\end{equation}
This means that two upper and two lower components of $\Psi$  transform independently under Lorentz transformation. We decoupled the 4-component spinor in the follow form for later convenience.
\footnote{
  some books's convention is
  \begin{equation}
     \Psi = \mqty(\admat{\psi_L \\ \psi_R})
  \end{equation}
  For the reason that $\mqty(\admat{\psi_L \\ 0})$ satisfied \eqref{definition of left chiral spinor} in Weyl representation is left chiral. So is the right-chiral part. It doesn't mean that $\psi_L$ and $\psi_R$ is chiral in two component representation
  }
\begin{equation}\label{decouple of Weyl representation}
  \Psi = \mqty(\admat{\chi \\ \h^*})
\end{equation}
$*$ denote that the lower one is inequivalent to the upper one. And we will see in fact the lower one is equivalent to the complex conjugate of the upper one.
From \eqref{definition of generators in Weyl representation} for upper two component, the generator is
\begin{equation} \label{the generator for  two component}
i\s^{ij} = -\frac{i}{4}\comm{\s^{i}}{\s^{j}} = \half\e_{ijk} \s^k ,\quad
i\s^{0i} = \frac{i}{2}\s^{i}
\end{equation}
from \eqref{Pauli matrice}, we see that $i\s^{ij}$ is hermitian while $i\s^{0i}$ is antihermitian. For lower two component
\begin{equation}
  i\bar{\s}^{ij} = -\frac{i}{4}\comm{\s^{i}}{\s^{j}} = \half\e_{ijk} \s^k ,\quad
i\bar{\s}^{0i} = -\frac{i}{2}\s^{i}
\end{equation}
Using  \eqref{parameter redefinition}\eqref{the generator for  two component} and \eqref{group element for weyl representation}
we can write the Lorentz transformation for two-component spinor (in infinitesimal form)
\begin{align}\label{infinitesimal transformation for two component spinor}
\de\chi &= (\frac{i}{2} \w_{ij}\half\e_{ijk}\s^k
       +i \w_{0i}\frac{i}{2}\s^{i})\chi
     = \frac{i}{2}\vec\s\cdot(\vec\tht+i\vec\f)\chi\notag\\
\de\h^* &= (\frac{i}{2} \w_{ij}\half\e_{ijk}\s^k
-i \w_{0i}\frac{i}{2}\s^{i})\h^*
= \frac{i}{2}\vec\s\cdot(\vec\tht-i\vec\f)\h^*
\end{align}

From the relation above we can immediately write the finite transformation matrice \eqref{definition of generators in Weyl representation} as
\begin{equation} \label{rewrite the transformation matrice of spinor representations}
S(\w) = \mqty(
s &
0                           \\
0                           &
{s^{-1}}^{\dagger})
\end{equation}
In analogue to Dirac Representation, we want to find the dual part of two-component spinor in order to construct Lorentz invariant scalar. For upper two-component spinor, using \eqref{infinitesimal transformation for two component spinor} and the property of Pauli matrice ${\s^{2}}^{\text{T}}=-\s^{2}$ and ${\s^{i}}^{\text{T}}\s^{2} = -\s^{2}\s^{i}$, which can be calculate straightfully by Pauli matrice
we find the Lorentz transformation of $(i\s^2\chi)^{\text{T}}$
\begin{equation} \label{transformation of dual part of chi}
\de{(i\s^2\chi)^{\text{T}}}
=-\frac{i}{2} (\tht_i + i \f_i)  (i\s^2 \chi)^{\text{T}} \s^{i}
\end{equation}
\eqref{infinitesimal transformation for two component spinor} and \eqref{transformation of dual part of chi} tell us that
$(i\s^2\chi)^{\text{T}}\chi$ is a scalar under Lorentz transformation. Then $i\s_2\chi$ can be the dual part of $\chi$.
If we denote $\chi$ by lower index
\begin{equation}
  \chi = \mqty(\chi_1 \\ \chi_2)
\end{equation}
then we denote its dual part by upper index
\begin{equation}
  i\s^{2}\chi = \mqty(0 & 1 \\ -1 & 0)
\mqty(\chi_1 \\ \chi_2)
=\mqty(\chi_2 \\ -\chi_1)
\equiv \mqty(\chi^{1}\\\chi^{2})
\end{equation}
now the scalar can be rewritten as $\chi^{\alpha} \chi_{\alpha}$
with $\chi^{\alpha} =  \e^{\a\be}\chi_{\be}$. Now $\e^{\a\be} =i\s^{2} $ is antisymmetric matrice playing the role of "metric tensor". The inversed relation is easily obtained $\chi_{\a} = \e_{\a\be}\chi^{\be}$ with the inverse matrice.
\begin{equation}\label{definition of undotted antisymmetric tensor}
  \e^{\a\be} = \mqty(0 & 1 \\ -1 & 0),\quad
\e_{\a\be} =\mqty(0 & -1 \\ 1 & 0)
\end{equation}
From \eqref{transformation of dual part of chi}, we get the finite transformation for $\chi^{\a}$, compare with the transformation for $\chi_{\a}$
\begin{equation} \label{Lorentz transformation for chi}
{\chi'}_{\a} = {(s)_{\a}}^{\be}\chi_{\be}
,\quad
{\chi'}^{\a} = {({s^{-1}}^{\text{T}})^{\a}}_{\be}\chi^{\be}
\end{equation}
$s$ is defined from \eqref{rewrite the transformation matrice of spinor representations}. If we write the explicit form for $s$ and
${s^{-1}}^{\tra}$ using \eqref{infinitesimal transformation for two component spinor}

\begin{equation} \label{finite transformation for upper two component spinor}
s =
\exp{\frac{i}{2}\vec{\s}\cdot(\vec{\tht}+i\vec{\f})},
\quad
{s^{-1}}^{\tra} = \exp{-\frac{i}{2}\vec{\s}^{\tra}\cdot(\vec{\tht}+i\vec{\f})}
\end{equation}
Using the property for pauli matrice that $\s^2{\s^{i}}^{\tra}\s^2 =- \s^i$ and $(\s^2)^2 =1$ we find that
\begin{equation} \label{equivalence of upper spinor with its dual space}
\s^2(s^{-1})^{\tra}\s^2 =\s^2(s^{-1})^{\tra}(\s^2)^{-1} =s
\end{equation}
means that $\chi^\a$ and $\chi_\a$ are in the equivalent representation.

In the same way, we study the lower two-component spinor. First the dual part of $\lh$ is $(-i\s^{2}\lh)^{\tra}$, since
\begin{equation} \label{infinitesimal transformation of dual h}
\de(-i\s^{2}\lh)^{\tra}
= -\frac{i}{2}(\tht_i-i\f_i)(-i\s^2\lh)^{\tra}\s^{i}
\end{equation}
From \eqref{infinitesimal transformation for two component spinor} and \eqref{infinitesimal transformation of dual h}, We see that
$(-i\s^{2}\lh)^{\tra}\lh$ transform as a scalar under Lorentz transformation. Thus $(-i\s^{2}\lh)$ can be a dual spinor of $\lh$. If we denote $\h$ by upper index
\begin{equation}
  \lh = \mqty({\lh}^{\dot{1}} \\ {\lh}^{\dot{2}})
\end{equation}
here dot on the index means the representation now is inequivalent from the upper two component one. The dual part is denoted by lower index.
\begin{equation}
  -i\s^2\lh = \mqty(0 & -1 \\ 1 & 0)
           \mqty(\lh^{\dot{1}}\\ \lh^{\dot{2}})
         = \mqty(-\lh^{\dot{2}}\\ \lh^{\dot{1}})
         = \mqty(\lh_{\dot{1}} \\ \lh_{\dot{2}})
\end{equation}

Now the Lorentz scalar for lower two-component spinor can be written as
\begin{equation}
  \lh_{\doa}\lh^{\doa}
= \e_{\doa\dob}\lh^{\dob}\lh^{\doa}
= \e^{\doa\dob}\lh_{\doa}\lh_{\dob}
\end{equation}
with the antisymmetric tensor defined as
\begin{equation}\label{definition of dotted antisymmetric tensor}
  \e^{\doa\dob} = \mqty(0 & 1 \\ -1 & 0),\quad
\e_{\doa\dob} =\mqty(0 & -1 \\ 1 & 0)
\end{equation}
From \eqref{rewrite the transformation matrice of spinor representations}, and the relation of \eqref{infinitesimal transformation for two component spinor} and \eqref{infinitesimal transformation of dual h}, we can write the finite transformation of lower two-component spinor under Lorentz transformation.
\begin{equation} \label{lorenta transformation for lh}
{\lh'}^{\doa} = {({s^{-1}}^{\dagger})^{\doa}}_{\dob}\lh^{\dob}
,\qquad
{\lh'}_{\doa} = {(s^{*})_{\doa}}^{\dob}\lh_{\dob}
\end{equation}
In analogue with \eqref{equivalence of upper spinor with its dual space}, in the same way we can have
\begin{equation} \label{equivalence of lower spinor with its dual space}
{s^{-1}}^{\dagger} = \s^2 s^* \s^2
\end{equation}
so $\lh^{\doa}$ and $\lh_{\doa}$ are equivalent representation.
Furthermore, by comparing \eqref{Lorentz transformation for chi} and \eqref{lorenta transformation for lh}, we are justified in identifying dotted spinors with complex conjugate of undotted ones, which is a convenient convention in Majorana condition.
\begin{equation}
  \lc_{\doa} = (\chi_{\a})^*
,\quad
\lc^{\doa} = (\chi^{\a})^*
\end{equation}

\begin{Exe}
Verify \eqref{transformation of dual part of chi} and \eqref{infinitesimal transformation of dual h}.
\end{Exe}

\begin{Exe}
Verify \eqref{equivalence of lower spinor with its dual space}.
\end{Exe}

\section{$SL(2,C)$ Group}

In this section we will see a deep relation between $SL(2,C)$ and the connected part of homogeneous Lorentz group, which enable us to write the vector representation in a new way.

To see these, let us first study more details on two-component spinor representation. In analogue to the analysis in Dirac representation, we are now in the position to derive its connection with group element in vector representation. From \eqref{unified form of gamma matrice} and \eqref{rewrite the transformation matrice of spinor representations}, \eqref{Lorentz transformation of gamma matrice} becomes

\begin{equation}
\mqty( 0 & \s^{\m} \\ \bar{\s}^{\m} & 0)
{\Lambda_{\m}}^{\rh}
= \mqty(s^{-1} & 0 \\ 0 & s^{\dagger} )
\mqty( 0 & \s^{\rh} \\ \bar{\s}^{\rh} & 0)
 \mqty(s & 0 \\ 0 & {s^{-1}}^{\dagger} )
\end{equation}
compare each element of matrice we have
\begin{equation} \label{Lorentz transformation of sigma matrice}
{\Lambda^{\rh}}_{\m} \s^{\m} =  s\s^{\rh} {s}^{\dagger}
,\qquad
{\Lambda^{\rh}}_{\m} \bs^{\m} =  {s^{-1}}^{\dagger}\bs^{\rh} s^{-1}
\end{equation}
From \eqref{Lorentz transformation for chi} and \eqref{lorenta transformation for lh}, the indices on $s$ and $s^{-1}$ are undotted, while those on $s^{\dagger}$ and ${s^{\dagger}}^{-1}$ are dotted, thus the indice on $\s^{\m}$ and $\s^{\m}$ are
\begin{equation}
  \s^{\mu} = (\s^{\m})_{\a\dob},\quad
\bs^{\rh} = (\bs^{\rh})^{\doa\be}
\end{equation}
For example, \eqref{Lorentz transformation of sigma matrice} now are
\begin{align}
{\Lambda^{\rh}}_{\m} (\s^{\m})_{\a\dob} &=
({s)_{\a}}^{\g}
(\s^{\rh})_{\g\dod}
{({s}^{\dagger})^{\dod}}_{\dob}\\
{\Lambda^{\rh}}_{\m} (\bs^{\m})^{\doa\be}
&=
{({s^{-1}}^{\dagger})^{\doa}}_{\dod}
(\bs^{\rh})^{\dod\g}
{(s^{-1})_{\g}}^{\be}
\end{align}
We can also use antisymmetric tensor to raise and lower the index as a definition. For example
\begin{equation}
  (\bs^{\mu})^{\doa\a} = \e^{\a\be}\e^{\doa\dob}(\s^{\m})_{\be\dob}
\end{equation}
There are huge amounts of relations of $\s$ and $\bs$ which are listed in Appendix.

A much interesting observation from \eqref{Lorentz transformation of sigma matrice} is that if we act a vector $V_{\rho}$ to both sides of the first equation, we find
\begin{equation} \label{Lorentz transformation for matrix representaton}
  V'_{\m}\s^{\m} = s V_{\rh}\s^{\rh}{s}^{\dagger}
\end{equation}
If we consider the object $v\equiv V_\m \s^{\m}$ as a new representation of Lorentz group. Then \eqref{Lorentz transformation for matrix representaton}, describes its transformation under Lorentz transformation $V_{\mu}\rightarrow V'_{\mu}$.
Furthermore, the number of bases in acting space is the same in both representation, in fact, you can find one to one correspondence in group acting space by considering
\begin{equation} \label{reflection to the vector representation}
\half \text{tr}[v\bs^{\m}] = V^{\m}
\end{equation}
these two representations is more likely to be equivalent.
We are now in the position to give the definition of $SL(2,C)$ group. Write the explicit form of $v$
\begin{equation}
  v = \mqty(-V_0 +  V_3 & V_1 - i V_2
                \\ V_1 +i V_2 & -V_0 -  V_3)
\end{equation}
Since $V_{\m}$ is an arbitrary real four-component vector.  $v$ is now an arbitrary hermitian matrix. Thus the transformation must be in the following form, to transform from one hermitian matrix into another.
\begin{equation} \label{transformation of hermitian space}
v' = \lambda v \lambda^{\dagger}
\end{equation}
which is consistent to \eqref{Lorentz transformation for matrix representaton}. Calculate the determinant
\begin{equation}
  \text{det}(v) = V_\m V^\m
\end{equation}
To satisfy the Lorentz invariant condition, we need $\text{det}(v)$ to be invariant, thus gives the condition for transformation matrice.
\begin{equation}
  |\text{det}(\lambda)| = 1
\end{equation}
since two $\lambda$s with a difference of phase give the same effect from \eqref{transformation of hermitian space}, we can conveniently adjust the phase so that
\begin{equation} \label{SL(2,C) condition}
\text{det}(\lambda) = 1
\end{equation}
The $2\times2$ complex matrices with unit determinant form a group, known as \kwd{$SL(2,C)$}. The group elements depend on 3 complex parameter, so the degree of freedom is the same as homogeneous Lorentz group. Now we need to ask whether $\lambda$ is the same as $s$, the group element of upper two-component spinor.

Any complex non-singular matrice $\la$ may be written in the form
\begin{equation}
    \la = e^{w}
\end{equation}
 \eqref{SL(2,C) condition} gives $\text{tr}(w) = 0$. So we can write $w$ in the basis of Pauli matrices $w = (a_i+i b_i)\s^{i}$ , where $a_i$ and $b_i$ are arbitrary real parameters. since
\begin{equation}
    V_0 = -\half\text{tr}(v)
\end{equation}
is invariant under $\la = e^{i b_i\s^{i}}$. So $b_i$s are the three parameters for spacetime rotations, the other three parameters $a_i$s should be the parameters for boosts in consistent with \eqref{finite transformation for upper two component spinor}, which shows $\lambda$ is the same as $s$. So the group space of $SL(2,C)$ and the homogeneous Lorentz group has a corresponding, however since $\la$ and $-\la$ has the same effect to the transformation $v$ from \eqref{transformation of hermitian space}. The relation is a double covering. The homogeneous Lorentz group is the same as $SL(2,C)/Z_2$.

From the above analysis, we conclude that $v$ is a representation of Lorentz group equivalent to vector representation, since we find the one to one corresponding of their acting space. And we can also choose a one to one corresponding for their transformation matrice. Thus makes them equivalent in the mapping perspective.

Furthermore from the transformation \eqref{Lorentz transformation for matrix representaton} and using the transformation law \eqref{Lorentz transformation for chi} and \eqref{lorenta transformation for lh}, we see that $v$ may be construct from two-component spinor
\begin{equation}
  v = \chi_\a \lc_{\dob}=\chi_\a(\chi_{\be})^*
=\mqty(\chi_1\chi_1^* & \chi_2\chi_1^*
  \\\chi_1\chi_2^* & \chi_2\chi_2^*)
\end{equation}
which is hermitian as required. So we may consider $v$ as a direct product of upper two-component and lower-two component spinor.

\begin{Exe}
Using the property of \eqref{eq:pauli-matrices-1}, verify \eqref{reflection to the vector representation}.
\end{Exe}




%======================= (A,B) Representation ======================%
\section{$(A,B)$ Representation}

We are now going to develop a classification of representation for homogeneous Lorentz group.



We know that the generator for homogeneous Lorentz group form a closed Lie Algebra \eqref{eq:arbitrary-homogeneous-Lie-algebra-commutation-1}, \eqref{eq:arbitrary-homogeneous-Lie-algebra-commutation-2}, \eqref{eq:arbitrary-homogeneous-Lie-algebra-commutation-3}
\begin{align}
&[K_i,K_j] = -i \e_{ijk} J_k
\\
&[J_i,K_j] = i \e_{ijk} K_k
\\
&[J_i,J_j] = i \e_{ijk} J_k
\end{align}
If we define new generators $\vec{A}, \vec{B}$
\begin{equation}\label{definition of A and B}
  \vec{A} \equiv \half(\vec{J}+i\vec{K}),\quad
\vec{B} \equiv \half(\vec{J}-i\vec{K})
\end{equation}
New commutation relation is
\begin{equation} \label{commutation relation of A and B}
\comm{A_i}{A_j} = i \e_{ijk} A_k,\quad
\comm{B_i}{B_j} = i \e_{ijk} B_k,\quad
\comm{A_i}{B_j} = 0
\end{equation}
The generators now are complexified and decoupled. We see $\vec{A}$ and $\vec{B}$ generate a group of $SU(2)$. This tells us that complexified Lorentz group can be construct by the direct product of two $SU(2)$ group. We know that the all the irreducible representation of $SU(2)$ can be classified by spin(integer or half-integer number). So now we can classify the representation of Lorentz group into two spin number (A,B). We denote the representation of spin-j for three SU(2) generators as $\vec{J}^{(j)}$

For $j = 0 , \half , 1$, we can have 
\begin{equation}\label{spin representation matrice for j = 0,1/2,1}
\vec{J}^{(0)} = 0, \qquad
\vec{J}^{(\half)} = \half \vec{\s},\qquad
(\vec{J}^{(1)}_i)_{jk} = i\e_{ijk}
\end{equation}
 
When $(A,B)= (0,0)$, then the lorentz generator $\vec{J}=\vec{K}=0$,  corresponds to scalar representation. 
When $(A,B)=(0,\half)$, which means
\begin{equation}\label{A,B value for (0,1/2)}
    \vec{A} = 0, \qquad \vec{B} = \half \vec{\s} 
\end{equation}
From \eqref{definition of A and B}, we have
\begin{equation}
    \vec{J} = \half \vec{\s},\qquad \vec{K}=  \frac{i}{2}\vec{\s}
\end{equation}
From  \eqref{finite transformation for upper two component spinor}
the infinitesimal form of s is
\begin{equation}
 s = 1 + \frac{i}{2} \vec{\s}\cdot(\vec{\tht}+i\vec{\f})
\end{equation} 
We see $(0,\half)$ is the upper-two component representation.
When $(A,B)=(\half,0)$
\begin{equation}\label{A,B value for (1/2,0)}
    \vec{A}= \half \vec{\s}, \qquad \vec{B} = 0 
\end{equation}
gives
\begin{equation}
    \vec{J} = \half \vec{\s}, \qquad \vec{K} =  -\frac{i}{2}\vec{\s}
\end{equation}
From \eqref{lorenta transformation for lh}, the infinitesimal form of lorentz transformation for the lower-two component representation is
\begin{equation}
{s^{-1}}^{\dagger} = 1 + \frac{i}{2}\vec{\s}\cdot(\vec{\tht}-i\vec\f)
\end{equation}
So we find $(\half,0)$ is the lower-two component representation.

From spin-1 representation of generators, we can see it is a rotation of three dimension vector. we will see $(0,1)$ and $(1,0)$ representation in later discussion.

Now I want to develop the definition of $(A,B)$ representation in an operator form. Since in quantum mechanics, all the field like component For example $O$ is an operator.and the Lorentz transformation can be written in the following form.
\begin{equation}
U(\Lambda)^{-1} O U(\Lambda) = M O
\end{equation}
with M furnish a representation of the homogeneous Lorentz group
\footnote{
	$U(\Lambda)$ is in fact the lorentz transformation of a physical state
	\begin{equation}
		\ket{\Omega} \rightarrow U(\Lambda) \ket{\Omega} \notag
    \end{equation}
	Thus, quantum mechanically $U(\Lambda)$ must be unitary (sometimes antiunitary if the flow of time direction is changed ) to preserve the possibility. 
	The expectation value of an symmetry operator changes by
	\begin{equation} 
	\bra{\Omega}O\ket{\Omega}\rightarrow \bra{\Omega} U^{-1}(\Lambda)O U(\Lambda)\ket{\Omega} \notag
	\end{equation}
	so we can write the transformation for such symmetry operator as
	\begin{equation}
	O \rightarrow  U^{-1}(\Lambda)O U(\lambda)\notag
	\end{equation}
}.
Since $U(\Lambda)$ is the symmetry operator for state vector, it must be unitary. The infinitesimal form of $U(\Lambda)$ is
\begin{equation} \label{infinitesimal form of ULa}
U(\Lambda) = 1 +  i\half \omega_{\mu\nu} \mathcal{J}^{\mu\nu}
=1 +  i \theta_k \mathcal{J}_k + i \f_i \mathcal{K}_i
\end{equation}
We use Floral letters to denote the generator acting on physical state, which must be Hermitian. Definition of $\mathcal{A}$ and$\mathcal{B}$ is similar with \eqref{definition of A and B}. 
Adjusting the physical state to be the eigenvector of the third component of $SU(2)$ generator, the definition of spin representation as follow.
\begin{equation} \label{the definition of spin representation}
\comm{\vec{\mathcal{A}}}{O_{ab}^{AB}} = -\sum_{a'}\vec{\mathcal{J}}^{(A)}_{aa'}O^{AB}_{a'b},\quad
\comm{\vec{\mathcal{B}}}{O_{ab}^{AB}} = -\sum_{b'}\vec{\mathcal{J}}^{(B)}_{bb'}O^{AB}_{ab'}
\end{equation}
with $a$ and $b$ run by unit steps from $-A$ to $+A$ and from $-B$ to $+B$. where $\vec{\mathcal{J}}^{j}$ is the spin three-vector  matrice for angular momentum $j$
\begin{equation} \label{three vector matrice for angular momentum}
(\mathcal{J}_1^{(j)}\pm \mathcal{J}_2^{(j)})_{\s'\s} = \de_{\s',\s\pm 1}
\sqrt{(j\mp\s)(j\pm\s+1)},\quad
(\mathcal{J}_3^{j})_{\s'\s} = \de_{\s'\s} \s
\end{equation}
Do the straight calculation we will see
\begin{equation}
  \vec{\mathcal{J}}^{(0)} = 0, \qquad
  \vec{\mathcal{J}}^{(\half)} = \half \vec{\s}
\end{equation}
Consider a scalar operator which transform under Lorentz transformation is
\begin{equation}
  U(\Lambda)^{-1} \F U(\Lambda) = \F
\end{equation}
Write it in the infinitesimal form, it's easy to see that $\F$ commutes with all $\mathcal{K}$ and $\mathcal{J}$, so that  $\F$ commutes with all $\mathcal{A}$ and $\mathcal{B}$. Thus according to \eqref{the definition of spin representation} and \eqref{three vector matrice for angular momentum}. We see the scalar representation belongs to $(0,0)$.

Consider the upper two-component spinor operator $\chi$ with transformation
\begin{equation} \label{Lorentz transformation of chi in operator form}
U(\Lambda)^{-1}\chi U(\Lambda) = s \chi
\end{equation}

Using \eqref{infinitesimal form of ULa}, and compare the coefficient of $\tht$ and $\f$ on both sides of \eqref{Lorentz transformation of chi in operator form} we have
\begin{equation}
  \comm{\vec{\mathcal{J}}}{\chi} = -\half \vec{\s}\chi,\qquad
\comm{\vec{\mathcal{K}}}{\chi} = - i\half \vec{\s}\chi
\end{equation}
or equivalently
\begin{equation}
  \comm{\vec{\mathcal{B}}}{\chi} =- \half \vec{\s}\chi,\qquad
\comm{\vec{\mathcal{A}}}{\chi} =0
\end{equation}
So upper two-component spinor $\chi$ is in the representation of $(0,\half)$, which is consistent to what we have illustrated before.

In the same way for the lower two-component spinor $\lh$ with transfromation
\begin{equation} \label{Lorentz transformation of lh in operator form}
U(\Lambda)^{-1}\lh U(\Lambda) = {s^{-1}}^{\dagger} \lh
\end{equation}
with \eqref{infinitesimal form of ULa} pluge into \eqref{Lorentz transformation of lh in operator form}, we have
\begin{equation}
  \comm{\vec{\mathcal{J}}}{\lh}=-\half\vec{\s}\lh,\quad
\comm{\vec{\mathcal{K}}}{\lh} = \frac{i}{2}\vec{\s}\lh
\end{equation}
or equivalently
\begin{equation}
  \comm{\vec{\mathcal{A}}}{\lh} = -\half \vec{\s}\lh,\qquad
\comm{\vec{\mathcal{B}}}{\lh} =0
\end{equation}
Thus lower two-component spinor $\lh$ is in representation of $(\half,0)$, also consistent in what we illustrated before.
Using the fact that vector representation is equivalent to the direct product of $(\half,0)$ and $(0,\half)$, we can concluded that the vector representation is in $(\half,\half)$. We can also verify this by using \eqref{the definition of spin representation}.

From \eqref{decouple of Weyl representation}, we can easily obtain that the Weyl representation is a direct sum of $(0,\half)$ and $(\half,0)$. 

%So far, we have developped all the useful representations in the later supersymmetric theory. Especially in the construction of supersymmetric algebra, keep in mind that all the formula should keep in the same pace while transform under Lorentz transformation, then it will be much easier to understand the form of each equation.

\begin{Exe}
Verify \eqref{commutation relation of A and B}.
\end{Exe}

%======================Direct Product Decomposition=================%
\section{Direct Product Decomposition}

So far, we have touched one direct product of representation of group element as well as group acting space is 
$(\half,0) \otimes (0,\half) \simeq (\half,\half)$. The generator of the direct product of two representations is a direct sum of the generators of origin two representation. We see from \eqref{A,B value for (0,1/2)} and \eqref{A,B value for (1/2,0)}. We can write
\begin{equation}
    \vec{A}_{(\half,\half)} 
    = \vec{A}_{(\half,0)} + \vec{A}_{(0,\half)}
    = \half{\vec{\s}}
\end{equation} 
\begin{equation}
    \vec{B}_{(\half,\half)} 
  = \vec{B}_{(\half,0)} + \vec{B}_{(0,\half)}
  = \half{\vec{\s}}
\end{equation}
Immediately, tells us that the vector representation is $(\half,\half)$.
In general, the direct product of some irreducible can not be irreducible again, because the generator calculated by direct sum can not be fit into a irreducible representation. For example, for $(\half,0)\otimes(\half,0)$
is not irreducible since $\vec{A}_{(\half,0)} + \vec{A}_{(\half,0)}
= {\vec{\s}}$ which can not satisfy Lie Algebra for $SU(2)$. 
But it can decoupled into the direct sum of several irreducible representation. The general decomposition rule takes form 
\begin{equation}
  O^{AB}_{ab} \otimes O^{CD}_{cd}
  =\sum_{E=|A-C|}^{A+C}
   \sum_{F=|B-D|}^{B+D}
   \sum_{e=-E}^{E}
   \sum_{e=-F}^{F}
   \times
   C_{AC}(E,e;a,c)
   C_{BD}(F,f;b,d)
   O^{EF}_{ef}
\end{equation}
 $O^{AB}_{ab}$ is an operator with spin $(A,B)$ where $a$ and $b$ run by unit steps from $-A$ to $A$ and from $-B$ to $B$. $C_{AB}(j,\s;a,b)$ is the usual \kwd{Clebsch Gordan} coefficient for coupling spins $A$ and $B$ to form spin j. From decomposition formula above, we can see the decomsition component can invlove all the spin $(E,F)$ with $\E$ from $|A-C|$ to $|A+C|$ and $F$ from $|B-D|$ to $B+D$. 
 For example, the direct product of vector representation 
 \begin{equation}
 (\half,\half)\otimes(\half,\half) \simeq (0,0)\oplus(1,0)\oplus(0,1)\oplus(1,1)
 \end{equation} 
 $(0,0)$ is a scalar construct by invariant tensor $g_{\mu\nu}V^{\m}V^{\n}$, with one degree of freedom.
 $(1,1)$ is a traceless symmetric part with 9 degree of freedom.
 $(1,0)$ and $(0,1)$ is self-dual or anti self-dual part for the antisymmetiric part with three degree of freedom respectively. Thus we see the degree of freedom is consistent.
 
 Now let us tell something about self-dual and anti self-dual representation.
 An antisymmetric representation from direct product of pair vectors representation is written as
 \begin{equation}
    V^{[\m,\n]} = V^{\m}V^{\n}-V^{\n}V^{\m} \equiv F^{\mu\nu}
 \end{equation} 
 The last step is because of the antisymmetric part from the direct product of pair vectors transform the same way as a convariant electromagnetic field strength.
 The dual part of $F^{\mu\nu}$ is defined by\footnote{We define the 4 dimension total antisymmetric matrice with $\e^{0123}$ = 1}
 \begin{equation}\label{definition of the dual part field strength}
   (*F)^{\mu\nu} = \frac{i}{2}\e^{\mu\nu\rho\s} F_{\rho\s}
 \end{equation}
 It is easy to verify that $*(*F) = F$.
 A(An) (anti)self-dual $(F^{\pm})^{\mu\nu}$ is defined as
 \begin{equation}\label{definition of self-dual and antiselfdual}
   *(F^{\pm})^{\mu\nu} = \pm (F^{\pm})^{\mu\nu}
 \end{equation}  
 thus we can immediately write
 \begin{equation}
    (F^{\pm})^{\mu\nu} = F^{\mu\nu} \pm (*F)^{\mu\nu}
 \end{equation}
 Since $F^{\mu\nu}$ has six degrees of freedom, we can devide it into electric part and magnetic part as usual convention by defining
 \begin{equation}
  E_k =\half \e_{ijk} F_{ij} \qquad B_i = F_{0i}
 \end{equation}
 and use \eqref{definition of the dual part field strength} and \eqref{definition of self-dual and antiselfdual}. We write the field strength component
 \begin{equation}
   F = (\vec{E},\vec{B}), \quad *F = (i\vec{B},-i\vec{E}), \quad
   F^{+} = ( \vec{E} + i\vec{B} , \vec{B} -i\vec{E}), \quad
   F^{-} = ( \vec{E} - i\vec{B} , \vec{B} +i\vec{E})
   \end{equation}
  Thus, we find $F^{\pm}$ has only independent variables $\vec{E}\pm i \vec{B}$ repectively. We now write the infinitesimal lorentz transformation without proof. Under rotation, the electric vector and magnetic vector transform as a usual three dimension vector.
  \begin{equation}
  \de_J\vec{E} = \vec{\w} \times \vec{E},\qquad 
  \de_J\vec{B} = \vec{\w} \times \vec{B}
  \end{equation}
  So for self-dual field strength $F^{+}$ ,we have
  \begin{equation}
  \de_J(\vec{E} + i\vec{B}) = \vec{\w} \times (\vec{E}+i\vec{B}) 
  \end{equation}
  The generators of rotation for $F^{+}$ is then
  \begin{equation}
  (J_i^{+})_{jk} = i\e_{ijk}
  \end{equation} 
  Under boost, the electric vector and magnetic vector is change into each other.
  \begin{equation}
  \de_B\vec{E} = \vec{\be} \times \vec{B}, \qquad
  \de_B\vec{B} = -\vec{\be} \times \vec{E}
  \end{equation}
  we have
  \begin{equation}
  \de_B(\vec{E} + i\vec{B}) =
  \vec{\be} \times (\vec{B}-i\vec{E})=
  -i\vec{\be}\times (\vec{E}+i\vec{B})
  \end{equation} 
  Thus the generators of boost for $F^{+}$ is given by
  \begin{equation}
  (K^{+}_i)_{jk} = \e_{ijk}
  \end{equation} 
  Calculate the $A$ and $B$ generator.
  \begin{equation}
  (A_i^{+})_{jk} = i\e_{ijk},\qquad 
  (B_i^{+})_{jk} = 0
  \end{equation} 
  compare with \eqref{spin representation matrice for j = 0,1/2,1}
  We find the self-dual representation $F^{+}$ is $(1,0)$ representation.
  In the same way we can have anti self-dual representation $F^{-}$ is $(0,1)$ representation.
  
  \begin{Exe}
  	Verify anti self-dual representation $F^{-}$ is $(0,1)$ representation.
  \end{Exe}
 %============================Majorana Spinor=========================
\section{Majorana Spinor}

In this section, we will introduce Majorana Spinor, which plays an important rule in supersymmetric field theory construction.

First we need to introduce \kwd{charge conjugation matrice} in four-component spinor representation.Since the Clifford algebra has only the one irreducible representation in Dirac representation and $(\g^{\m})^{\tra}$ also satisfy Clifford algebra \eqref{defination of Dirac matrice 1}, it must be related by $\g^{\m}$ by a similarity transformation. 
\begin{equation}\label{definition of charge conjugate matrice}
  C (\g^{\m})^{\tra} C^{-1} = -\g^{\m}
\end{equation}
In Weyl representation we can write the charge conjugation matrice in following form.
\begin{equation}
C = i \g^2\g^0 
= i\mqty(-\s^{2} & 0    \\
          0      & \s^{2})
\end{equation}
$i$ is a phase convention choiced to make $C$ real. The explicit form of charge conjugation matrice immediately shows the property for $C$
\begin{equation}
C = - C^{\tra} = C^{*} = -C^{\dagger} = -C^{-1}
\end{equation}
The antiparticle $\Psi^{c}$ is defined as
\begin{equation}
\Psi^{c} = C \bar{\Psi}^{\tra}
\end{equation}
From \eqref{Lorentz transformation of barPsi}, we get the transformation for antiparticle spinor as
\begin{equation}
\Psi^{c} \rightarrow {\Psi^{c}}' = C {S^{-1}}^{\tra} {\bar\Psi}^{\tra}
 =S \Psi^{c}
\end{equation}
where we have used \eqref{definition of charge conjugate matrice} in calculation. So we know that the antiparticle spinor transforms in the same way as the usual spinor. If the particle is neutral, in the sence that it is its own antiparticle, we now write the \kwd{Majorana condition}.
\begin{equation}\label{Majorana condition}
 \Psi = \Psi^{c}
\end{equation}
We then write the Majorana spinor in decoupled way as in \eqref{decouple of Weyl representation} 
\begin{equation}
\Psi^{c} = i\mqty(-\s^{2} & 0    \\
           0      & \s^{2})
           \mqty(\h^{\a}  \\  \lc_{\doa})
         = \mqty(\h_{\a} \\ \lc^{\doa})
\end{equation}
where we have used the definition for antisymmetric tensor \eqref{definition of undotted antisymmetric tensor} and \eqref{definition of dotted antisymmetric tensor}. Using the Majorana condition, we find $\h = \chi$. So the Majorana spinor can be written in decoupled way as
\begin{equation}\label{decoupled form of majorana spinor}
\Psi^{m} = \mqty(\chi_{\a} \\ \lc^{\doa})
\end{equation}
So the Majorana condition half the degree of freedom of spinor. 
In supersymmetric theory we often construct a 4-component Majorana spinor from a 2-component spinor. \eqref{decoupled form of majorana spinor} tells us that, if we have a $(0,\half)$ spinor $u$, then the Majorana spinor can be construct as
\begin{equation}\label{construct the Majorana spinor from (0,half)}
\mqty(u \\ i\s^{2}u^*)
\end{equation}
If we have a $(\half,0)$ spinor $v$, then the Majorana 4-component spinor can be constructed as
\begin{equation}
\mqty(-i\s^{2}v^*\\v)
\end{equation}
From \eqref{construct the Majorana spinor from (0,half)}
The dual part of a Majorana spinor now is
\begin{equation}
\bar{\Psi}
=-\Psi^{\dagger}\g^{0}
= -\mqty(u \\ i\s^{2}u^*)^{\dagger}\g^{0}
=\mqty(u^{\tra} & (i\s^{2}u^{*})^{\tra})
\mqty(-i\s^2 & 0 \\
      0     & i\s^2)
=\Psi^{\tra}C 
\end{equation} 
We can also derive this relation using Majorana condition \eqref{Majorana condition}.

Now we need to study the the property of constant Majorana spinor, each component of the spinor should be a number instead of an operator. The number now should be Grassman number. If $a$ and $b$ are Grassman number, then
\begin{equation}
   ab = -ba 
\end{equation}
in the sense that the lorentz scalar $\chi^{\a}\chi_{\a}$ can be non-trivial. For a pair of Majorana spinors $\tht_1$ , $\tht_2$ and any $4\times4$ numerical matrice $M$ 
\begin{equation}
\bar{\tht}_1 M \tht_2 
= \sum_{\a\be} (\tht_{1})_{\a} (CM)_{\a\be}(\tht_2)_{\be}
= \sum_{\a\be} (\tht_2)_{\a} (M^{\tra}C)_{\a\be}(\tht_2)_{\be}
= \bar{\tht}_2 C^{-1} M^{\tra} C \tht_2
\end{equation}
As have mentioned that $4\times4$ complex matrices can be constructed by following sixteen independent matrix $1,\g^{\m},\Si^{\mu\nu},\g^{\m}\g^5,\g^5$, let $M$ be these bases, using the definition for charge conjugation matrix \eqref{definition of charge conjugate matrice}, we can obtain
\begin{equation}\label{property of charge conjugation matrix C}
  M^{\tra} = 
  \begin{cases}
     C M C^{-1}   \qquad  M = 1, \g^{\m}\g^5, \g^5 \\
     -C M C^{-1}   \qquad M = \g^{\m}, \Si^{\mu\nu} 
\end{cases}
\end{equation}
It follows that 
\begin{equation}
\bar{\tht}_1 M \tht_2 =
  \begin{cases}
    \bar{\tht}_2 M \tht_1   \qquad  M = 1, \g^{\m}\g^5, \g^5 \\
    -\bar{\tht}_2 M \tht_1    \qquad M = \g^{\m}, \Si^{\mu\nu}
  \end{cases}
\end{equation}
In particular, if $\tht_1 = \tht_2$,  the bilinear of $M = \g^{\m}, \Si^{\mu\nu}$ vanishes.
\begin{equation}\label{the constraint on bilinear of covariant matrices}
 \bar{\tht}\g^{\m}\tht = 0,\qquad \bar{\tht}\Si^{\mu\nu}\tht = 0
\end{equation} 
So the only covariant bilinears are $\bar{\tht}\tht$ , $\bar{\tht}\g^{5}\g^{\m}\tht$ and $\bar{\tht}\g^{5}\tht$. This limits help us rewrite the product of two Majorana spinors. The completeness of $16$ covariant matrices and  Lorentz invariance required that $\tht\bar{\tht}$ takes the form 
\begin{equation}
\tht\bar{\tht}=
k_S(\bar{\tht}\tht)+
k_V\g_{\m}(\bar{\tht}\g^{\m}\tht)+
K_T\Si_{\mu\nu}(\bar{\tht}\Si^{\mu\nu}\tht)+
k_A\g^5\g_{\m}(\bar{\tht}\g^5\g^{\m}\tht)+
k_P\g^5(\bar{\tht}\g^{5}\tht)
\end{equation}
The constraint \eqref{the constraint on bilinear of covariant matrices} show that we may take $k_V = 0$ and $k_T = 0$. The remaining coeffcients can be calculated by multiplying on the right with $1$, $\g_{5}\g^{\m}$ and $\g_{5}$ and taking the trace. From \eqref{orthogonal property of 16 covariant matrices}, we obtain $k_S = -\frac{1}{4}$, $k_A = \frac{1}{4}$ and $k_P = -\frac{1}{4}$. Then we have
\begin{equation}
\tht\bar{\tht}=
-\frac{1}{4}(\bar{\tht}\tht)+
\frac{1}{4}\g^5\g_{\m}(\bar{\tht}\g^5\g^{\m}\tht)
-\frac{1}{4}\g^5(\bar{\tht}\g^{5}\tht)
\end{equation}  
By multiplying on the right side with $C$, we have
\begin{equation}
\tht_{\a}\tht_{\be} = 
\frac{1}{4} (C)_{\a\be} (\bar\tht\tht)- 
\frac{1}{4}(\g^5\g_{\m}C)_{\a\be}(\bar{\tht}\g^5\g^{\m}\tht)+
\frac{1}{4}(\g^5C)_{\a\be}(\bar{\tht}\g^{5}\tht)
\end{equation}
If we define $C = \g^5\e = \e\g^5$ with 
\begin{equation}
\e = \mqty(-i\s^2 &  0 \\
             0    & -i\s^2)
\end{equation}
then 
\begin{equation}
\tht_{\a}\tht_{\be} = 
\frac{1}{4} (\e\g^5)_{\a\be} (\tht^{\tra}\e\g^5\tht)+ 
\frac{1}{4}(\g_{\m}\e)_{\a\be}({\tht}^{\tra}\e\g^{\m}\tht)+
\frac{1}{4}(\e)_{\a\be}(\tht^{\tra}\e\tht)
\end{equation}

Now consider the product of $\tht_{\a}\tht_{\be}\tht_{\g}$. We can divide the $\tht$ into left- and right-handed parts $\tht_L$ and $\tht_R$ with the definition \eqref{definition of left chiral spinor} and \eqref{definition of right chiral spinor}. Since Each of $\tht_L$ and $\tht_R$ has only two independent components, we have
$(\tht_L)_{\a}(\tht_L)_{\be}(\tht_L)_{\g} = 0$. For all $\a,\be,\g$ and therefore
\begin{equation}
\tht_\a \tht_\be \tht_\g = 
(\tht_L)_\a(\tht_L)_\be(\tht_R)_\g+
(\tht_L)_\a(\tht_R)_\be(\tht_L)_\g+
(\tht_R)_\a(\tht_L)_\be(\tht_L)_\g
+ L\leftrightarrow R
\end{equation}  

\begin{Exe}
	Verify \eqref{property of charge conjugation matrix C}.
\end{Exe}

